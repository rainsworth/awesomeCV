\cvsection{Volunteer Work}
\begin{cventries}
\cventry
    {Organiser}
    {HER+Data MCR}
    {Manchester, UK}
    {July 2017 - present}
    {
      \begin{cvitems}
        \item {Established the Manchester chapter of HER+Data, a meetup group to bring together women who work with and love data - to support one another, inspire each other, share experiences and talk data (\url{https://meetup.com/HER-Data-MCR}).}
        \item {Organise monthly cross-city events and meetups: lead event planning and promotion, coordinate with sponsors, venues and speakers, and work with and manage a team of co-organisers.}
        \item {Identify new partners to work with, create opportunities for collaboration and support, and connect tech companies, industry stakeholders and academic institutions with our 670+ members to cultivate a community that actively supports women in STEM.}
        \item {Responsible for content creation and management of social media (\url{https://twitter.com/herplusdatamcr}).}
      \end{cvitems}
    }
\cventry
    {Expert}
    {Mozilla Open Leaders, Mozilla Foundation}
    {Manchester, UK}
    {September 2018 - present}
    {
      \begin{cvitems}
        \item{Invited to act as Expert during Rounds 6 and 7 of Mozilla's Open Leadership Training.}
        \item{The vision of Mozilla Open Leaders is to strengthen open projects and communities around the world. Experts play a vital role supporting new leaders as they practice working open and facilitating connections within the network.}
        \item{During the program, any of the mentors or organisers may contact me to share my expertise during either a 30 minute online meeting with a participant in the program to talk about their specific project, or a 90 minute online meeting with the entire cohort.}
        \item {Facilitated a 3-day Gallery Session of Pulsar Hunters at Mozilla Festival 2018: an interactive Open Science installation where participants could learn in person about pulsar signals from Jodrell Bank researchers and contribute to classifying real data taken with the LOFAR Telescope.}
      \end{cvitems}
    }
\cventry
    {Mentor and Cohort Host}
    {}
    {}
    {February - May 2018}
    {
      \begin{cvitems}
        \item{Invited to act as Mentor and Cohort A Host during Round 5 of Mozilla's Open Leadership Training.}
        \item{Reviewed and rated Round 5 Project Lead applications.}
        \item{Met with my assigned Project Lead every two weeks to provide support and guidance as they went through the training and learned how to work openly on their projects.}
        \item{Hosted Full Cohort meetings (60 - 90min, 35 participants) every two weeks, which involved emceeing/moderating each of the calls for Cohort A, and providing support/setting the tone for a cohort to bond together as a group during the program.}
        \item{Gained valuable coaching skills and connections across the Mozilla and open source community.}
        \item{Hosted a local site and participated in the Global Sprint (May 10-11, 2018, \url{http://bit.ly/MozSprintMCR}).}
      \end{cvitems}
    }
\cventry
    {Open Project Lead}
    {}
    {}
    {September - December 2017}
    {
      \begin{cvitems}
        \item{Selected for Round 4 of Mozilla's Open Leadership program where I received training and developed skills in open leadership, project management, building communities, open communications and running awesome events. (''Open'' refers to working collaboratively, making outputs freely accessible for others to use in order to maximise impact, and promoting inclusion, equity and diversity.)}
        \item{Founded \textit{Resources for Open Science in Astronomy} (\url{https://github.com/rainsworth/ROSA}), an ongoing program to enable a more collaborative and open culture in academia (\url{https://bit.ly/ROSAinterview}).}
        \item {Designed and hosted a 1 hour session at Mozilla Festival 2017 to engage 20 participants in a discussion on how to best advocate Open Science to researchers (\url{https://bit.ly/MozFestRecap}).}
        \item {Strong communicator to a range of stakeholders and audiences (researchers, advocates, policy makers, general public) through interviews, blog posts, social media, podcasts, online project demonstrations and fostering our cohort's Twitter community: \#RebelFoxes.}
      \end{cvitems}
    }    
\end{cventries}
